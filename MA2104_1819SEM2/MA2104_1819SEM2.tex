\documentclass{article}

\usepackage{mathtools}
\usepackage[utf8]{inputenc}
\usepackage{amsmath}
\usepackage{amssymb}
\usepackage{graphicx}
\usepackage{bm}
\usepackage{enumitem}
\usepackage{textcomp}
\DeclarePairedDelimiter\ceil{\lceil}{\rceil}
\DeclarePairedDelimiter\floor{\lfloor}{\rfloor}

\usepackage[margin=1in]{geometry}
\setlength\parindent{0pt}
\setlength{\parskip}{5px}

\newtheorem{lemma}{Lemma}

\newcommand\at[2]{\left.#1\right|_{#2}}
\newcommand{\p}[2]{\frac{\partial #1}{\partial #2}}
\newcommand{\abs}[1]{\left| #1 \right|}
\newcommand{\paren}[1]{\left(#1\right)}
\newcommand{\brac}[1]{\left[#1\right]}
\newcommand{\R}{\mathbb{R}}
\newcommand{\angbrac}[1]{\left\langle #1 \right\rangle}

\begin{document}

\subsection*{Q1}
\begin{enumerate}[label=(\alph*)]
\item
$\ell: \begin{pmatrix}
1\\2\\1
\end{pmatrix}
k, k \in \R
$

Let $v = \begin{pmatrix}
3\\3\\3
\end{pmatrix}$

Projection of $v$ to $\ell$: $\begin{pmatrix}
3\\3\\3
\end{pmatrix}\cdot \abs{\begin{pmatrix}
1\\2\\1
\end{pmatrix}}^{-1}\begin{pmatrix}
1\\2\\1
\end{pmatrix}
= \frac{12}{\sqrt{6}}
$

Perpendicular vector of $v$ to $\ell$: $\begin{pmatrix}
3\\3\\3
\end{pmatrix} - \frac{12}{\sqrt{6}} \abs{\begin{pmatrix}
1\\2\\1
\end{pmatrix}}^{-1}\begin{pmatrix}
1\\2\\1
\end{pmatrix}
= \begin{pmatrix}
1\\-1\\1
\end{pmatrix}
$

Distance: $\abs{\begin{pmatrix}
1\\-1\\1
\end{pmatrix}} = \sqrt{3}$

\item
\begin{enumerate}[label=(\roman*)]
\item
Let $f(u,v) = \begin{pmatrix}
u\\v\\u^2-v^2+3
\end{pmatrix}$

$f_u = \begin{pmatrix}
1\\0\\2u
\end{pmatrix}, f_v = \begin{pmatrix}
0\\1\\-2v
\end{pmatrix}$

$f_u(P) = \begin{pmatrix}
1\\0\\6
\end{pmatrix}, f_v(P) = \begin{pmatrix}
0\\1\\-6
\end{pmatrix}$

$\pi: \begin{pmatrix}
3\\3\\3
\end{pmatrix} + \lambda \begin{pmatrix}
1\\0\\6
\end{pmatrix} + \mu \begin{pmatrix}
0\\1\\-6
\end{pmatrix}$

\item
To find the intersection, we solve the equation: $
\begin{pmatrix}
u\\v\\u^2-v^2+3
\end{pmatrix}
=
\begin{pmatrix}
3\\3\\3
\end{pmatrix} + \lambda \begin{pmatrix}
1\\0\\6
\end{pmatrix} + \mu \begin{pmatrix}
0\\1\\-6
\end{pmatrix}
$

$u=3 + \lambda, v = 3+\mu$

$u^2 - v^2 + 3 = (3 + \lambda)^2 - (3+\mu)^2 + 3 = 3 + 6\lambda - 6\mu$

$\therefore \lambda^2 - \mu^2 = 0$

Hence, the line $\ell_1: x=y,z=3$ corresponds to $\lambda=\mu$, which satisfies the equation above. Hence, it lies in the intersection of $S$ and $\pi$.

\item
The other line would be $\lambda = -\mu$.

$\ell_2: \begin{pmatrix}
3\\3\\3
\end{pmatrix} + \lambda\begin{pmatrix}
1\\-1\\12
\end{pmatrix}$
\end{enumerate}

\end{enumerate}

\subsection*{Q2}

\begin{enumerate}[label=(\alph*)]
\item
$f(x,y) = x^3 - y^3 + 2xy$

$f_x(x,y) = 3x^2 + 3y, f_y(x,y) = -3y^2 + 3x$

To find critical points, we solve for $f_x = f_y = 0$.

$3x^2 + 3y = -3y^2 + 3x = 0$ only at $(0,0)$ and $(1,-1)$

$(0,0)$ is a saddle point, because if we fix $y=0$ and vary $x$, we can see that $f$ increases as $x$ increases and decreases as $x$ decreases.

$f_{xx} = 6x, f_{yy} = -6y, f_{xy} = 3$. $D(1,-1) = 36 - 9 > 0$.

Hence, $(1,-1)$ is a local min.

\item
$g(x) = x^2 + 2xy + 2y^2$

$\nabla g (x) = \begin{pmatrix}
2x+2y\\2x+4y
\end{pmatrix}
$, which is non zero under the condition $g(x)=5$.

$\nabla f(x) = \begin{pmatrix}
2\\1
\end{pmatrix}
$

By Lagranch multiplier, $\begin{pmatrix}
2\\1
\end{pmatrix} = \lambda \begin{pmatrix}
2x+2y\\2x+4y
\end{pmatrix} \implies 2\lambda(2x+4y) = \lambda (2x+2y) \implies x=-3y$

Sub $x=-3y$ to $x^2 + 2xy + 2y^2=5$: $9y^2 + 2(-3y)y+2y^2 = 5 \implies y^2=1$

$\therefore (x,y) = (-3,1)$ or $(3,-1)$.

$f(-3,1)=-5, f(3,-1) = 5$.

Min is $-5$, max is $5$.

\end{enumerate}

\subsection*{Q3}
\begin{enumerate}[label=(\alph*)]
\item
Let $
\begin{pmatrix}
u\\v
\end{pmatrix}
= \begin{pmatrix}
x+y\\y-2x
\end{pmatrix}
$

Then A, B, C would have $u-v$ coordinates $(0,0),(3,-6),(3,3)$. Let the region be $S$.

$
\p{(u,v)}{(x,y)} = \abs{\begin{pmatrix}
\p{u}{x} & \p{u}{y} \\
\p{v}{x} & \p{v}{y} \\
\end{pmatrix}} = \abs{\begin{pmatrix}
1 & 1\\
-2 & 1
\end{pmatrix}} = 3
$

\begin{align*}
\iint_R \sqrt{x+y} (y-2x)^2\ dx\ dy &= \iint_S uv^2 \frac{1}{3}\ dv\ du \\
&= \int_0^3 \int_{-2u}^u \frac{1}{3} uv^2\ dv\ du \\
&= \int_0^3 \brac{\frac{1}{9}uv^3}_{-2u}^u\ du \\
&= \int_0^3 u^4\ du \\
&= \brac{\frac{1}{5}u^5}_0^3 \\
&= \frac{3^5}{5}
\end{align*}

\item
$f(x,y,z) = xy\sin z$ is a potential function for $F$.

\[
\int_C F\cdot dr = f\paren{\frac{\pi}{2},\frac{\pi}{2},\frac{\pi}{2}} - f(0,0,0) = \frac{\pi^2}{4}
\]

\end{enumerate}

\subsection*{Q4}


\begin{enumerate}[label=(\alph*)]
\item
$P = 7y - e^{\sin x}$, $Q = 9x - \cos(y^3+7y)$
\begin{align*}
\int P\ dx + Q\ dy &= \iint_D \p{Q}{x} - \p{P}{y}\ dA \\
&= \iint_D 9-7\ dA \\
&= 2 (\pi \times 2^2) \\
&= 8\pi
\end{align*}
\item
Cone: $\sqrt{3}z = \sqrt{x^2+y^2}$

Sphere: $x^2+y^2+(z-1)^2 = 1$

Let $r^2 = x^2 + y^2$, and $x = r\cos \theta, y = r\sin \theta$.

First, we find the intersection of the sphere and cone:

\[
z = \frac{\sqrt{x^2+y^2}}{\sqrt{3}} = 1+\sqrt{1-(x^2+y^2)}
\]
\[
\frac{r}{\sqrt{3}} = 1 + \sqrt{1-r^2}
\]
\[
1-r^2 = \paren{\frac{r}{\sqrt{3}} - 1}^2 = \frac{r^2}{3} - \frac{2r}{\sqrt{3}} + 1
\]
\[
\frac{4}{3}r^2 - \frac{2}{\sqrt{3}}r = 0 \implies r=\frac{\sqrt{3}}{2}
\]
Hence, the volume is:
\begin{align*}
\int_0^{2\pi} \int_0^{\sqrt{3}/2} \int_{r/\sqrt{3}}^{1+\sqrt{1-r^2}} r\ dr\ d\theta
&= \int_0^{2\pi} \int_0^{\sqrt{3}/2} \paren{1+\sqrt{1-r^2} - \frac{r}{\sqrt{3}}} r\ dr\ d\theta \\
&= 2\pi \brac{-\frac{1}{3}(1-r^2)^{3/2} + \frac{1}{2}r^2 - \frac{1}{3\sqrt{3}}r^3}_0^{\sqrt{3}/2} \\
&= \frac{13}{12}\pi
\end{align*}
\end{enumerate}

\subsection*{Q5}

\begin{enumerate}[label=(\alph*)]
\item
The original conditions given are:
\[-1 \leq x \leq 1\]
\[x^2 \leq y \leq 1\]
\[0 \leq z \leq 1-y\]
Hence,
\[0\leq z\leq 1-y \leq 1-x^2 \leq 1\]
\[x^2 \leq y \leq 1-z \implies -\sqrt{1-z} \leq x \leq \sqrt{1-z}\]
\[x^2 \leq y \leq 1-z\]
Hence,
\[
\int_{-1}^1 \int_{x^2}^1 \int_0^{1-y} f(x,y,z)\ dz\ dy\ dx
= \int_{0}^1 \int_{-\sqrt{1-z}}^{\sqrt{1-z}} \int_{x^2}^{1-z} f(x,y,z)\ dy\ dx\ dz
\]

\item
\[
r(\theta) = \begin{pmatrix}
-\sin \theta\\\cos\theta\\\cos^2 \theta-\sin^2 \theta
\end{pmatrix}, 0\leq \theta \leq 2\pi
\]

\begin{align*}
\int_C F\cdot dr &= \int_0^{2\pi} F(r(\theta))\cdot r'(\theta)\ d\theta \\
&= \int_0^{2\pi} \begin{pmatrix}
\sin^3 \theta\\\cos\theta\\\sin^3\theta\cos^3\theta
\end{pmatrix}\cdot \begin{pmatrix}
-\sin \theta\\\cos\theta\\\cos^2 \theta-\sin^2 \theta
\end{pmatrix}\ d\theta \\
&= \int_0^{2\pi} -\sin^4 \theta + \cos^2\theta + \sin^3\theta\cos^5\theta - \sin^5\theta\cos^3\theta\ d\theta \\
&= \int_0^{2\pi} -\sin^4 \theta + \cos^2\theta \ d\theta \\
&= \frac{\pi}{4}
\end{align*}
\end{enumerate}

\subsection*{Q6}

\begin{enumerate}[label=(\alph*)]
\item
\begin{align*}
\int_C f\nabla g\cdot dr &= \iint_\Sigma \nabla \times (f\nabla g)\cdot d\Sigma \\
&= \iint_\Sigma \nabla \times \begin{pmatrix}
fg_x\\fg_y\\fg_z
\end{pmatrix}
\cdot d\Sigma \\
&= \iint_\Sigma \brac{\paren{\p{}{y}(fg_z)-\p{}{z}(fg_y)}i - \paren{\p{}{x}(fg_z)-\p{}{z}(fg_x)}j + \paren{\p{}{x}(fg_y)-\p{}{y}(fg_x)}k }\cdot d\Sigma \\
&= \iint_\Sigma \brac{\paren{f_yg_z-f_zg_y}i - \paren{f_xg_z-f_zg_x}j + \paren{f_xg_y-f_yg_x}k }\cdot d\Sigma \\
\end{align*}
Similarly,
\[
\int_C g\nabla f\cdot dr = \iint_\Sigma \brac{\paren{g_yf_z-g_zf_y}i - \paren{g_xf_z-g_zf_x}j + \paren{g_xf_y-g_yf_x}k }\cdot d\Sigma
\]
Hence,
\[
\int_C f\nabla g\cdot dr = -\int_C g\nabla f\cdot dr = \int_{-C} g\nabla f\cdot dr
\]

\item

Let
\[G(x,y,z) = \angbrac{\frac{x}{(x^2+y^2+z^2)^{3/2}},\frac{y}{(x^2+y^2+z^2)^{3/2}},\frac{z}{(x^2+y^2+z^2)^{3/2}}} = \frac{r}{|r|^3}\]
\[H(x,y,z) = \angbrac{0,0,z^2}\]
Then, $F(x,y,z) = G(x,y,z) + H(x,y,z)$

Note that
\[\nabla \cdot G = 0 \text{ for } r\neq 0\]
\[\nabla \cdot H = 2z\]

Let $E$ be the solid ellipsoid $x^2+\frac{y^2}{4}+\frac{z^2}{9}=1$.

\begin{align*}
\int_S H\cdot dS
&= \iiint_E (\nabla\cdot H)\ dV \\
&= \iiint_E 2z\ dV \\
&= 0 & \text{Due to symmetry}
\end{align*}

Let $D$ be the solid ellipsoid minus the sphere centered at $(0,0,0)$ with radius $r_0 = 1/2$, and let $C$ be the surface of that sphere.
\[
\int_S G\cdot dS - \int_C G\cdot dS = \iiint_D(\nabla\cdot G)\ dV = 0
\]
\[\therefore \int_S G\cdot dS = \int_C G\cdot dS\]
\[
\int_C G\cdot dS = \int_C (G\cdot n)\ dS = \int_C \paren{\frac{r}{|r|^3}\cdot n}\ dS = \int_C \frac{1}{|r|^2}\ dS = 4\pi{r_0}^2 \frac{1}{{r_0}^2} = 4\pi
\]
Hence,
\[
\int_S F\cdot dS = \int_S H\cdot dS + \int_S G\cdot dS = 4\pi
\]

\end{enumerate}

\end{document}
