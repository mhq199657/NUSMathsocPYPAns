\PassOptionsToPackage{svgnames}{xcolor}
\documentclass[12pt]{article}



\usepackage[margin=1in]{geometry}  
\usepackage{graphicx}             
\usepackage{amsmath}              
\usepackage{amsfonts}              
\usepackage{framed}   
\usepackage{mathtools}            
\usepackage{amssymb}
\usepackage{array}
\usepackage{amsthm}
\usepackage[nottoc]{tocbibind}
\usepackage{bm}
\usepackage{enumitem}


\DeclareMathOperator{\Tr}{Tr}
 \newcommand{\im}{\mathrm{i}}
  \newcommand{\diff}{\mathrm{d}}
  \newcommand{\col}{\mathrm{Col}}
  \newcommand{\row}{\mathrm{R}}
  \newcommand{\kerne}{\mathrm{Ker}}
  \newcommand{\nul}{\mathrm{Null}}
  \newcommand{\nullity}{\mathrm{nullity}}
  \newcommand{\rank}{\mathrm{rank}}
  \newcommand{\Hom}{\mathrm{Hom}}
  \newcommand{\id}{\mathrm{id}}
  \newcommand{\ima}{\mathrm{Im}}
  \newcommand{\lcm}{\mathrm{lcm}}
  \newcommand{\diag}{\mathrm{diag}}
  \newcommand{\expec}{\mathrm{E}}
  \newcommand{\var}{\mathrm{var}}
  \newcommand{\cov}{\mathrm{cov}}
  \newcommand{\inv}{^{-1}}
  \newcommand{\str}{^\ast}
  \newcommand{\Poi}{\mathrm{Poisson}}
  \newcommand\norm[1]{\left\lVert#1\right\rVert}
\setlength{\parindent}{0cm}
\setlength{\parskip}{0em}
\newcommand{\Lim}[1]{\raisebox{0.5ex}{\scalebox{0.8}{$\displaystyle \lim_{#1}\;$}}}
\newtheorem{definition}{Definition}[section]
\newtheorem{theorem}{Theorem}[section]
\newtheorem{notation}{Notation}[section]
\theoremstyle{definition}
\DeclareMathOperator{\arcsec}{arcsec}
\DeclareMathOperator{\arccot}{arccot}
\DeclareMathOperator{\arccsc}{arccsc}
\DeclareMathOperator{\spn}{Span}
\setcounter{tocdepth}{1}
\begin{document}

\title{PYP Answer - MA2216 AY1617Sem2}
\author{Ma Hongqiang}
\maketitle
\begin{enumerate}
  \item \begin{enumerate}
  \item We can get the cdf of $X$ by integrating $f_X(x)$:
  \[
F_X(x)=\begin{cases}
0&x\leq -1\\
\frac{1}{4}(1+x)^2&-1<x<1\\
1&x\geq 1
\end{cases}
  \]
  Since $Y=X^2$, we see that $F_Y(y)=P(Y<y)=P(X^2<y)$. Therefore,
  \[
F_Y(y)=\begin{cases}
0&y\leq 0\\
P(-\sqrt{y}<x<\sqrt{y})=\frac{1}{4}(1+\sqrt{y})^2-\frac{1}{4}(1-\sqrt{y})^2=\sqrt{y}&0<y<1\\
1&y\geq 1
\end{cases}
  \]
  By differentiating the cdf of $Y$, we have
  \[
f_Y(y)=\begin{cases}
\frac{1}{2}y^{-\frac{1}{2}}&0<y<1\\
0&\text{elsewhere}
\end{cases}
  \]
  \item As $y$ vanishes in region other than $[0,1]$, we can evaluate $\expec(Y)$ and $\var(Y)$ in $[0,1]$ only.
  \[
\expec(Y)=\int_0^1 yf_Y(y)\diff y = \int_0^1 \frac{1}{2}y^\frac{1}{2}\diff y = \left[\frac{1}{3}y^\frac{3}{2}\right]_0^1 = \frac{1}{3}
  \]
  And similarly,
  \[
\var(Y) = \int_0^1(y^2-\expec(Y)^2)f_Y(y)\diff y = \int_0^1(y^2-\frac{1}{9})\frac{1}{2}y^{-\frac{1}{2}}\diff y = \frac{4}{45}
  \]
        \end{enumerate}
\item\begin{enumerate}
\item Since $Y\sim\Poi(n)$, we can express $Y = \sum_{i=1}^nY_i$, where $Y_i\sim\Poi(1)$ for $i=1,\ldots, n$. We note $Y_i$ has mean $\mu=1$ and variance $\sigma^2=1$.\\When $n$ is large, apply Central Limit Theorem on $Y_i,i=1,\ldots, n$, we have
\[
\frac{\sum_{i=1}^n Y_i-n(1)}{\sqrt{1}\sqrt{n}}=\frac{Y-n}{\sqrt{n}}\sim Z(0,1)
\]
which implies $P(Y<y)\approx \Phi(\frac{y-n}{\sqrt{n}})$.\\
By continuity correction, \[P(Y=n)=P(Y<n+\frac{1}{2})-P(Y<n-\frac{1}{2})=\Phi(\frac{1}{2\sqrt{n}})-\Phi(\frac{-1}{2\sqrt{n}})\;\;\;(\#)\]. 
\item On one hand, $P(Y=n)=\frac{e^{-n}n^n}{n!}$, since $Y\sim\Poi(n)$. \\On the other hand, $P(Y=n)$ can be evaluated using (\#). When $n$ is large, $\frac{1}{2\sqrt{n}}$ is small, so the RHS of (\#) becomes the area of rectangle on the pdf of $Z(0,1)$ distribution with base length $(\frac{1}{2\sqrt{n}}-(-\frac{1}{2\sqrt{n}}))=\frac{1}{\sqrt{n}}$ and height given by the pdf $\frac{1}{\sqrt{2\pi}}$.Therefore, we have
\[
\frac{\left(\frac{n}{e}\right)^n}{n!}=\frac{1}{\sqrt{2\pi n}}
\]
Rearranging the above equation implies the result. 
\end{enumerate}
\item\begin{enumerate}
\item $\expec(X_1)=0, \var(X_1)=1, \cov(X_1,X_2)=\rho$.
\item We calculate $\expec(Z)$ by conditioning on $X_1$:
\begin{align*}
\expec(Z)&=\expec(\expec[Z|X_1])\\
&=\expec(Z|X_1<X_2)P(X_1<X_2)+\expec(Z|X_1\geq X_2)P(X_1\geq X_2)\\
&=\expec(X_2)P(X_1<X_2)+\expec(X_1)P(X_1\geq X_2)
\end{align*}
Next, calculate $\expec(Y)$ by conditioning also on $X_1$:
\begin{align*}
\expec(Y)&=\expec(\expec[Y|X_1])\\
&=\expec(Y|X_1<X_2)P(X_1<X_2)+\expec(Y|X_1\geq X_2)P(X_1\geq X_2)\\
&=\expec(X_1)P(X_1<X_2)+\expec(X_2)P(X_1\geq X_2)
\end{align*}
Adding the two equations above, we have
\begin{align*}
\expec(Y)+\expec(Z)&=\expec(X_1)(P(X_1<X_2)+P(X_1\geq X_2))+\expec(X_2)(P(X_1<X_2)+P(X_1\geq X_2))\\&=\expec(X_1)+\expec(X_2)
\end{align*}
\item We recognise that $X_1$ follows a normal distribution with mean 0.\\ By Chebyshev's inequality,
\[
P(|X_1-0|\geq 3)\leq \frac{1^2}{3^2}=\frac{1}{9}
\]
Therefore, $P(X_1\geq 3)\leq \frac{1}{18}$, by symmetry of normal distribution.
\item We have, from the joint density,
\begin{align*}
\mathbb{P}\left\{\min(X_1,X_2) \leq y\right\} &= 1 - \mathbb{P}\left\{ X_1 > y, X_2 > y \right\} \\
&= 1 - \int_{y}^{\infty} \int_y^{\infty} f_{X_1,X_2}(s,t)dsdt \\
&= 1 -\int_y^{\infty} f_{X_2}(t)\int_y^{\infty} f_{X_1|X_2}(s,t) ds dt
\end{align*}
where 
\[
f_{X_2} (t)=\frac{1}{\sqrt{2\pi}}e^{-\frac{t^2}{2}}, \qquad f_{X_1|X_2}(s,t)=\frac{1}{\sqrt{2\pi(1-\rho^2)}}e^{-\frac{(s-\rho t)^2}{2(1-\rho^2)}}
\]
and so
\begin{align*}
\mathbb{P}\left\{\min(X_1,X_2) \leq y\right\} &= 1 - \int_{y}^{\infty} \varphi(t) \left(1 - \Phi\left( \frac{y -\rho t}{\sqrt{1-\rho^2}}\right) \right) dt. \\
&= 1 - \int_{y}^{\infty} \varphi(t) \Phi\left( \frac{\rho t - y }{\sqrt{1-\rho^2}} \right) dt
\end{align*}
To get the density we differentiate with respect to $y$ giving
\begin{align*}
f_{Y}(y) &= -\frac{\partial}{\partial y}\int_{y}^{\infty}\varphi(t)\Phi\left(\frac{\rho t - y}{\sqrt{1-\rho^2}}\right) dt \\
&= \varphi(y)\Phi\left(\frac{\rho y - y}{\sqrt{1-\rho^2}}\right) + \int_{y}^{\infty}\varphi(t) \frac{1}{\sqrt{1-\rho^2}}\frac{1}{\sqrt{2\pi}}e^{-\frac{(\rho t - y)^2}{2(1-\rho^2)}}dt \tag{1}
\end{align*}
Completing the square of the last term in the above equation we have
\begin{align*}
\varphi(t)\frac{1}{\sqrt{2\pi(1-\rho^2)}}e^{-\frac{(\rho t-y)^2}{2(1-\rho^2)}} &= \frac{1}{\sqrt{2\pi}}\frac{1}{\sqrt{2\pi(1-\rho^2)}}e^{-\frac{1}{2(1-\rho^2)}\left((1-\rho^2)t^2 +(\rho t -y)^2  \right)} \\
&=\frac{1}{\sqrt{2\pi}}\frac{1}{\sqrt{2\pi(1-\rho^2)}}e^{-\frac{1}{2(1-\rho^2)}\left(t^2 - 2 t \rho y + \rho^2 y^2 + (1-\rho^2)y^2  \right)} \\
&= \frac{1}{\sqrt{2\pi}}e^{-\frac{y^2}{2}}\frac{1}{\sqrt{2\pi(1-\rho^2)}}e^{-\frac{(t-\rho y)^2}{2(1-\rho^2)}}.
\end{align*}
So putting this back in to (1) we get
\begin{align*}
f_{Y}(y) &= \varphi(y)\Phi\left(\frac{\rho y -y}{\sqrt{1-\rho^2}}\right) + \varphi(y)\int_y^{\infty} \frac{1}{\sqrt{2\pi(1-\rho^2)}}e^{-\frac{(t-\rho y)^2}{2(1-\rho^2)}}dt \\
&= \varphi(y)\Phi\left(\frac{\rho y -y}{\sqrt{1-\rho^2}}\right) +  \varphi(y)\left( 1 - \Phi\left(\frac{y - \rho y}{\sqrt{1-\rho^2}}\right)\right) \\
&= 2\varphi(y)\Phi\left(\frac{\rho y -y}{\sqrt{1-\rho^2}}\right).
\end{align*}
\end{enumerate}
\item \begin{enumerate}
  \item We note that
  \[
P(X=x)=\frac{\binom{m}{x}\binom{N-m}{n-x}}{\binom{N}{n}}
  \]
  and
  \[
P(X=x-1)=\frac{\binom{m}{x-1}\binom{N-m}{n-x+1}}{\binom{N}{n}}
  \]
  So
  \begin{align*}
\frac{P(X=x)}{P(X=x-1)}&=\frac{\frac{m!}{x!(m-x)!}\frac{(N-m)!}{(N-m-n+x)!(n-x)!}}{\frac{m!}{(x-1)!(m-x+1)!}\frac{(N-m)!}{(N-m-n+x-1)!(n-x+1)!}}\\
&=\frac{m-x+1}{x}\cdot\frac{n-x+1}{N-m-n+x}=(\#)\\
  \end{align*}
  Similarly,
  \begin{align*}
\frac{P(Y=x)}{P(Y=x-1)}&=\frac{\binom{n}{x}\left(\frac{m}{n}\right)^x\left(\frac{N-m}{N}\right)^{n-x}}{\binom{n}{x-1}\left(\frac{m}{n}\right)^{x-1}\left(\frac{N-m}{N}\right)^{n-x+1}}\\
&=\frac{n-x+1}{x}\frac{m}{N-m}=(*)
  \end{align*}
  To show the equality of limit, write
  \begin{align*}
\lim_{m,N\to\infty}(\#)&=\frac{n-x+1}{x}\frac{1-\frac{x-1}{m}}{\frac{N}{m}-1-\frac{n-x}{m}}\\&=\frac{n-x+1}{x}\frac{1}{p-1}\\&=\frac{n-x+1}{x}\frac{m}{N-m}=(*)
  \end{align*}
  \item Let the number of cocainer packets be $m$. Then the scenario will have the following possibilities to occur
  \[
P(X=4)\times P(X=0)\approx\binom{4}{4}\left(\frac{m}{496}\right)^4\binom{2}{0}\left(\frac{492-(m-4)}{492}\right)
  \]
  By differentiating the expression to solve for maximum, we have $m=\frac{992}{3}$, which gives the number 0.0233 if substituted back.
  \item So $20-6 = 14$ more packets is sufficient. 
\end{enumerate}
\end{enumerate}
\end{document}